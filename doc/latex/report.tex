\documentclass[11pt]{article}
\usepackage{geometry}
\geometry{letterpaper}

\usepackage{graphicx}
\usepackage{amssymb, amsmath}
\usepackage{epstopdf}
% \usepackage{natbib}
\usepackage[hyperref=auto,style=alphabetic,backend=bibtex]{biblatex}
\usepackage[colorlinks]{hyperref}
\DeclareGraphicsRule{.tif}{png}{.png}{`convert #1 `dirname #1`/`basename #1 .tif`.png}
\addbibresource{report.bib}

%\title{Title}
%\author{Name 1, Name 2}
%\date{date}

\begin{document}


%!TEX root = report.tex
\thispagestyle{empty}

\begin{center}
\includegraphics[width=5cm]{ETHlogo.eps}

\bigskip


\bigskip


\bigskip


\LARGE{ Lecture with Computer Exercises:\\ }
\LARGE{ Modelling and Simulating Social Systems with MATLAB\\}

\bigskip

\bigskip

\small{Project Report}\\

\bigskip

\bigskip

\bigskip

\bigskip


\begin{tabular}{|c|}
\hline
\\
\textbf{\LARGE{Traffic simulation in the city of Z\"urich}}\\
\\
\hline
\end{tabular}
\bigskip

\bigskip

\bigskip

\LARGE{Jan D\"{o}rrie, Simone Forte, \\Charalampos Gkonos, Athina Korfiati}



\bigskip

\bigskip

\bigskip

\bigskip

\bigskip

\bigskip

\bigskip

\bigskip

Z\"urich\\
May 2014\\

\end{center}



\newpage

%%%%%%%%%%%%%%%%%%%%%%%%%%%%%%%%%%%%%%%%%%%%%%%%%

\newpage
\section*{Agreement for free-download}
\bigskip


\bigskip


\large We hereby agree to make our source code for this project freely available for download from the web pages of the SOMS chair.
Furthermore, we assure that all source code is written by ourselves and is not violating any copyright restrictions.

\begin{center}

\bigskip


\bigskip

Jan Wilken D\"{o}rrie \hfill Simone Forte \hfill Charalampos Gkonos \hfill Athina Korfiati

\end{center}
\newpage

%%%%%%%%%%%%%%%%%%%%%%%%%%%%%%%%%%%%%%%



% IMPORTANT
% you MUST include the ETH declaration of originality here; it is available for download on the course website or at http://www.ethz.ch/faculty/exams/plagiarism/index_EN; it can be printed as pdf and should be filled out in handwriting


%%%%%%%%%% Table of content %%%%%%%%%%%%%%%%%

\tableofcontents

\newpage

%%%%%%%%%%%%%%%%%%%%%%%%%%%%%%%%%%%%%%%



\section{Abstract}

\section{Individual contributions}

\section{Introduction and Motivations}

\section{Description of the Model}

\section{Implementation}
We started out by extracting the OpenStreetMap file that contains Zurich.
In order to do so we downloaded a recent Switzerland extract from Geofabrik \cite{geofab} and then extracted the boundaries of Zurich with the tool \emph{Osmconvert} \cite{osmconv}.
Afterwards we used another tool called \emph{Osmosis} to limit our data to the roads in Zurich.
Furthermore we modified the data using JOSM which enabled us to simplify the road network, making our algorithms more efficient.
Once we removed small isolated connected components resulting from clipping the data to Zurich we wrote an A-Star algorithm implementation in C++.
We considered doing this directly in Matlab, however the performance loss was too large, so that we wrote the mex file.

The MATLAB implementation mostly consists of two parts, an initialization part and an update part. In the initialization the main data structures and parameters of the model are set up; we thus generate the routes for every car, picking starting and ending location according to a probability distribution specified by the CSVs files (reference...) and the actual path will be generated by computing the shortest path between the two locations. In the update part, for every timestep, we compute the new locations of every car according to the way specified earlier in the Description of the Model section and we compute the new GEO coordinates of the cars which are then given back to the Javascript client.


Finally we made use of the Javascript Library Leaflet to visualize our results on the actual map of Zurich.
We enabled communication between MATLAB and Javascript using the included Java classes for Sockets.
The Javascript code issues HTTP GET requests telling the MATLAB implementation which methods to perform.
Afterwards the location of every car in Zurich was passed as an JSON object through the socket.


\section{Simulation Results and Discussion}

\section{Summary and Outlook}

% \section{References}

\newpage
\printbibliography


\end{document}




